%%%%%%%%%%%%%%%%%%%%%%%%%%%%%%%%%%%%%%%%%
% University/School Laboratory Report
% LaTeX Template
% Version 3.0 (4/2/13)
%
% This template has been downloaded from:
% http://www.LaTeXTemplates.com
%
% Original author:
% Linux and Unix Users Group at Virginia Tech Wiki 
% (https://vtluug.org/wiki/Example_LaTeX_chem_lab_report)
%
% License:
% CC BY-NC-SA 3.0 (http://creativecommons.org/licenses/by-nc-sa/3.0/)
%
%%%%%%%%%%%%%%%%%%%%%%%%%%%%%%%%%%%%%%%%%

%----------------------------------------------------------------------------------------
%	PACKAGES AND DOCUMENT CONFIGURATIONS
%----------------------------------------------------------------------------------------

\documentclass{article}

\usepackage{mhchem} % Package for chemical equation typesetting
\usepackage{siunitx} % Provides the \SI{}{} command for typesetting SI units
\usepackage{amsmath}
\usepackage{amssymb}
\usepackage{listings}
%\usepackage{lstcoq}
\usepackage{fullpage}
\usepackage{hyperref}
\usepackage{graphicx} % Required for the inclusion of images

\setlength\parindent{0pt} % Removes all indentation from paragraphs

\renewcommand{\labelenumi}{\alph{enumi}.} % Make numbering in the enumerate environment by letter rather than number (e.g. section 6)

%\usepackage{times} % Uncomment to use the Times New Roman font

%----------------------------------------------------------------------------------------
%	DOCUMENT INFORMATION
%----------------------------------------------------------------------------------------

\title{Project 1 - Report} % Title

\author{Cibele \textsc{Freire} and Theodore \textsc{Sudol}} % Author name

\date{\today} % Date for the report
\newcommand{\link}[3]{%
    \underline{\texttt{\href{#1:#2}{#3}}}%
}

\begin{document}

%\lstset{language=Coq} probably should set this to typescript

\maketitle % Insert the title, author and date

\begin{center}
\begin{tabular}{l r}
%Date Performed: & January 1, 2012 \\ % Date the experiment was performed
%Partners: & James Smith \\ % Partner names
%& Mary Smith \\
Course: CPMSCI 630 - Systems \\
Instructor: Professor Emery Berger % Instructor/supervisor

\end{tabular}
\end{center}



% If you wish to include an abstract, uncomment the lines below
%\begin{abstract}
%This document is In this document we describe how the development of this system was conducted. The choices that were made 
%\end{abstract}


\section{Introduction}\label{intro}

\paragraph{}
The project consists of designing and implementing an interpreter for Python bytecode in TypeScript/JavaScript. The interpreter is suppose to work in the browser, loading and executing programs from Python bytecode files, i.e. .pyc files.

\paragraph{}
In Section~\ref{parser} we explain how we parsed the bytecode, what is the Marshall format and what data structures we used. In Section~\ref{interpreter} we explain the design of the interpreter and what we were able to implement.

\section{Parser}\label{parser}

\paragraph{}
The bytecode generated by a Python compiler is stored in Marshall format. Each .pyc file is composed of:

\begin{itemize}
	\item Magic number: consist of the first 8 bytes of the .pyc file and it indicates from which Python version the bytecode was generated;
	\item Date: the next 8 bytes and it indicates the date of compilation;
	\item Code object: the remaining of the file. More details on Section~\ref{code object}
\end{itemize}

We define the class \texttt{Unmarshaller} to handle the parsing phase. The class has the attributes index, input, magicNumber, date, internedStrs, and output.

\begin{itemize}
	\item \texttt{index}: it keeps track of which byte of the file is being currently read 
	\item \texttt{input}: the path for the .pyc file that is going to be processed
	\item \texttt{magicNumber}: stores the magic number
	\item \texttt{date}: stores the date of compilation
	\item \texttt{internedStrs}:
	\item \texttt{output}: the code object generated after parsing the .pyc file
\end{itemize}

The following are the central methods implemented by this class.

\begin{itemize}
	\item \texttt{value()}:
	\item \texttt{unmarshallCodeString()}:
	\item \texttt{unmarshal()}:
\end{itemize}

\subsection{Code objects}\label{code object}

Code objects are the main ``object'' in a .pyc file, it contains the information you need to execute what the original python program was describing. We defined the class \texttt{Py\_CodeObject} to store the code object for a .pyc file.

\begin{itemize}
	\item \texttt{argcounts}: the number of arguments that the function/code requires
	\item \texttt{nlocals}: number of local variables
	\item \texttt{stacksize}: the depth of the stack that is going to be used by that code
	\item \texttt{flags}: %NOT SURE WHAT TO SAY HERE
	\item \texttt{code}: the code that will be executed. It consists of a sequence of bytecode instructions, each of which has an opcode associated to it. The list of instructions and their respective opcodes can be found in \texttt{Systems630-Project1/src/opcodes.ts} on the repository.
	\item \texttt{consts}: the constant objects. Those can ben numbers, strings, collections (such as lists, or dictionaries), and also code objects.  
	\item \texttt{names}: 
	\item \texttt{varnames}: the names of all the variables defined in the code object
	\item \texttt{freevars}, \texttt{cellvars}: %NOT SURE WHAT TO SAY HERE
	\item \texttt{filename}: the name of the file the originated the code object
	\item \texttt{name}: indicates if the code object refers to a function, and then it is the name of the function, or it will be \texttt{module}, indicating that it is the ``main'' code.
	\item \texttt{firstlineno}: the line number where this code object started on the .py that compiled into the .pyc file
	\item \texttt{lnotab}: %NOT SURE WHAT TO SAY HERE
\end{itemize}

The only method in this class is the constructor.

\section{Interpreter}\label{interpreter}

\texttt{Py\_FrameObject}

\texttt{Interpreter} 

Explain the frame execution

\subsection{Numeric methods}

Issues with numeric methods and how they were fixed.

\subsection{Using the browser}


\subsection{What was not implemented}




%\bibliographystyle{unsrt}
%\bibliography{sample}

%----------------------------------------------------------------------------------------


\end{document}